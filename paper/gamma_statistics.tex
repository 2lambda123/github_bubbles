\newpage
\section{Gamma distribution as a likelihood for smoothed data}
\lb{app:gamma}

In this appendix we show that smoothing the data and using gamma distribution instead of the Poisson distribution
is a reasonable procedure (at least in the case when the scale of variations in the diffuse emission is larger than the smoothing radius).
We split the log likelihood into a sum over areas with size approximately equal to the smoothing scale
and neglect the correlation between the different areas after smoothing the data.
Suppose that one of these areas has $n$ pixels with random variables $k_i$ drawn from a Poisson distribution with mean $\ld_0$
(here we assume that the underlying diffuse flux is approximately constant within the smoothing radius).
The likelihood function for parameter $\ld$ is
\be
L(\ld) = \prod_i \frac{\ld^{k_i}}{k_i !} e^{-\ld}.
\ee
The log likelihood is
\be
\log L = \sum_{i = 1}^n (-\ld + k_i \log \ld) + const.
\ee
The best-fit value $\ld_*$ is determined from
\be
0 = \frac{\p \log L}{\p \ld} = -n + \frac{1}{\ld} \sum_{i = 1}^n k_i,
\ee
which gives $\ld_* = \frac{1}{n} \sum_{i = 1}^n k_i$.
The uncertainty is
\be
\frac{1}{\sm^2} = - \left. \frac{\p^2 \log L}{\p \ld^2} \right|_{\ld = \ld_*} = \frac{n}{\ld_*},
\ee
which gives $\sm^2 = \ld_* / n$.

If the smoothing radius is comparable to the size of the region,
then we can approximate the values $k_i$ with the average in the region $\tilde{k}_i = \bar{k} = \frac{1}{n} \sum_{i = 1}^n k_i$
(note that $\tilde{k}_i$ are not integers).
We take the gamma distribution for the likelihood function 
\be
\tilde{L} = \prod_i \frac{\ld^{\td{k}_i}}{\G(\td{k}_i + 1)} e^{-\ld}.
\ee
The log likelihood is
\be
\log \td{L} = \sum_{i = 1}^n (-\ld + \td{k}_i \log \ld) + const.
\ee
The best-fit solution is the same as in the Poisson case $\td{\ld}_* = \bar{k} = {\ld}_*$.
The uncertainty is also the same as in the Poisson case:
\be
\frac{1}{\td{\sm}^2} = - \left. \frac{\p^2 \log \td{L}}{\p \ld^2} \right|_{\ld = \td{\ld}_*} = \frac{n}{\td{\ld}_*} = \frac{1}{\sm^2}.
\ee
Thus, smoothing the data and using the gamma distribution (in this case) gives the same result as using the original Poisson distribution,
which shows that the procedure is well defined from the statistical point of view and it gives reasonable results.