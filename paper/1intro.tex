\section{Introduction}
\lb{sec:intro}



The \Fermi bubbles (FBs) are one of the most spectacular and unexpected discoveries 
in the \Fermi Large Area Telescope (LAT) data \citep{2010ApJ...724.1044S}.
The FBs extend to $55^\circ$ above and below the Galactic center (GC),
they have a sharp edge and a relatively uniform intensity across the surface, apart from a ``cocoon'' in the south eastern part of the bubbles
\citep{2012ApJ...753...61S, 2014ApJ...793...64A}.
The spectrum is $\sim E^{-2}$ at GeV energies with a cutoff or a softening around 100 GeV at latitudes $|b| > 10^\circ$ \citep{2014ApJ...793...64A}.
The origin of the FBs is attributed either to an emission from the supermassive black hole (SMBH) at
the center of our Galaxy %(AGN scenario)
or to a period of starburst activity which resulted in a combined wind
from supernova (SN) explosions of massive stars %(starburst scenario),  
\citep[for a review see][]{2010ApJ...711..818S}.
The gamma-ray signal up to 100 GeV can be produced either by interactions of hadronic cosmic rays (CR) with gas (hadronic model)
or by inverse Compton (IC) scattering of high energy electrons and positrons and the interstellar radiation photons (leptonic model).

Although the FBs were detected about 8 years ago, their origin is still unresolved.
Important insights into their origin can be obtained from the study of the morphology and the spectrum of the FBs near the GC.
%Understanding the origin of the FBs will provide a unique opportunity to test the predictions of numerical simulations  of either the emission from the SMBH or the starburst activity near the GC.
%A study of the gamma-ray emission near the GC is important to understand the origin of the FBs.
The spectrum of gamma rays can provide information on the composition of the 
CR that produce the gamma-ray signal (hadronic vs leptonic CR),
as well as the age of the CR (through a cooling cutoff in the leptonic scenario or a break due to escape of high energy CR)
and the spectrum of the CR at the source.
The morphology of the emission can point to the source of the bubbles: either the SMBH Sgr A* or a recent star-forming region.
Preliminary analyses of the FBs at low latitudes showed indications of higher intensity of emission near the Galactic plane (GP) and a displacement
to negative longitudes \citep{2016ApJS..223...26A, 2017ApJ...840...43A},
the spectrum of the FBs for $|b| < 10^\circ$ is consistent with a power-law $\propto E^{-2}$ 
without a cutoff up to 1 TeV \citep{2017ApJ...840...43A}.
%If confirmed, the high intensity and the hard spectrum of the FBs at low latitudes will open up a possibility of a detection of the FBs with current and future imaging atmospheric Cherenkov telescopes (IACTs) and with neutrino telescopes, which will further constrain models of the FBs formation.

The study of the FBs in the GP is complicated due to bright Galactic diffuse emission components.
The $\pi^0$ component of the gamma-ray emission is well traced by the distribution of gas in most of the GP,
but it has large uncertainties towards the GC due to a lack of kinematic information from the motion of the gas in the 
Galaxy, which is used to reconstruct the gas distribution
(the velocity of the gas in the direction of the GC is perpendicular to the line of sight).
There are also large uncertainties in the distribution of the CR sources in the Galactic plane, especially near the GC,
which make it rather difficult to predict a priori 
%, even using the standard tracers, such as supernova remnants (SNRs), 
the distribution of the propagated CR in the Galaxy.
For the IC component of the gamma-ray emission, 
there is a significant uncertainty in the interstellar radiation field (ISRF) density near the GC \citep[e.g.,][]{2017ApJ...846...67P} in addition to 
the uncertainties in the CR distribution.
There should also be undetected point-like and extended sources, which nevertheless contribute to the total flux.
As a result, the agreement of the gamma-ray data with Galactic propagation models, such as GALPROP \citep{2007ARNPS..57..285S}, 
is rather poor in the GP from the statistical point of view, even after adjusting the components in energy bins and in Galactic rings.

In this paper, we analyze the gamma-ray emission at the base of the FBs and 
estimate the uncertainties %in the residual hard and bright component
%is an analysis of the FBs at low latitudes to estimate the uncertainties on the gamma-ray emission 
related to modeling of the Galactic foreground and background components.
We focus on morphology and spectrum of the FBs at energies from 10 GeV to 1 TeV,
where the relative intensity of the gamma-ray emission from the FBs is higher than at low energies due to softer spectra of the Galactic components.
The study of the FBs at high energies will be also important for future searches with neutrino and Cherenkov telescopes.

%which we subtract from the data to find the gamma-ray emission from the FBs.
%Since the Galactic gamma-ray emission has large uncertainties towards the GC \citep[e.g.,][]{Calore:2014xka, 2017ApJ...840...43A},
We use several methods to determine the Galactic foreground/background emission at high energies.
The goal is to bracket the uncertainty in the foreground and background emission in the determination of 
the gamma-ray flux from the FBs.
First, since the FBs emission shows an anisotropy relative to the GC
while the foreground emission is expected to be symmetric (up to asymmetries in the gas distribution),
as an estimate of the possible contribution of the FBs at low latitudes, 
we find the difference in the gamma-ray data to the east and to the west of the GC 
(after masking bright point sources).
Second, we use the fact that the spectrum of the FBs is harder than the spectra of the other components of Galactic diffuse emission
above $\sim$ 1 GeV.
Consequently, the contribution of the FBs at low energies near the GP is relatively small. 
We use the data below 1 GeV as a template of the Galactic foreground emission, 
which we fit to the data at energies above 1 GeV.
The residual emission is used to estimate the emission from the FBs.
Finally, we determine a model for the Galactic gamma-ray emission using the GALPROP code%
\footnote{\url{http://galprop.stanford.edu}} 
\citep{Moskalenko:1997gh, Strong:1998fr, Strong:2004de, Ptuskin:2005ax, 2007ARNPS..57..285S, Porter:2008ve,Vladimirov:2010aq}. 
In this model, we allow many free parameters, such as rescaling of the $\pi^0$ and bremsstrahlung emission in Galactocentric rings and
refitting of bright point sources near the GC.
Due to many free parameters, this model absorbs as much of the gamma-ray emission at the base of the FBs as it can.
Consequently, it gives us a lower bound on the emission from the tentative hard component at the base of the FBs.




