\section{Introduction}
\lb{sec:intro}



The \Fermi bubbles (FB) are one of the most spectacular and unexpected discoveries based on the \Fermi LAT data
\citep{2010ApJ...724.1044S}.
The FB extend to $55^\circ$ above and below the Galactic center (GC)
and have a hard spectrum with a cutoff or a softening around 100 GeV at latitudes $|b| > 10^\circ$ \citep{2014ApJ...793...64A}.
The origin of the FB is attributed either to an emission from the supermassive black hole (SMBH) at
the center of our Galaxy (AGN scenario)
or to a period of starburst activity which results in a combined wind
from supernovae explosions of massive stars (starburst scenario),  for a review see \cite{2010ApJ...711..818S}.

Although the FB were detected more than 7 years ago, their origin is still an unresolved question.
Important insights into their origin can be obtained from the study of the morphology and the spectrum of the FB near the GC.
%Understanding the origin of the FB will provide a unique opportunity to test the predictions of numerical simulations  of either the emission from the SMBH or the starburst activity near the GC.
%A study of the gamma-ray emission near the GC is important to understand the origin of the FB.
The spectrum of gamma-rays may provide information on the composition of the 
cosmic rays (CR) that produce the gamma-ray signal (hadronic vs leptonic CR),
as well as the age of the CR (through a cooling cutoff in the leptonic scenario or a break due to escape of high energy CR)
and the spectrum of the CR at the source.
Preliminary analyses of the FB at low latitudes show a higher intensity of emission near the GP and a displacement
to negative longitudes \citep{2016ApJS..223...26A, 2017ApJ...840...43A},
the spectrum of the FB for $|b| < 10^\circ$ is consistent with a power-law $\propto E^{-2}$ 
without a cutoff up to 1 TeV \citep{2017ApJ...840...43A}.
If confirmed, the high intensity and the hard spectrum
of the FB at low latitudes will open up a possibility of a detection of the FB with current and future imaging atmospheric Cherenkov telescopes (IACTs)
and with neutrino telescopes, which will further constrain the models of the FB formation.

The study of the FB in the Galactic plane is complicated due to bright Galactic diffuse emission components.
The $\pi^0$ component of gamma-ray emission is well traced by the distribution of the gas in most of the GP,
but it has large uncertainties towards the GC due to lack of kinematic information from the motion of the gas in the 
Galaxy (the velocity of the gas along the line of sight towards the GC is perpendicular to the line of sight),
which is used to reconstruct the distribution of the gas.
There are also large uncertainties in the distribution of the CR sources in the Galactic plane (and especially near the GC),
which make it rather difficult to predict a priori (even using the standard tracers, such as SNRs) the distribution
of CR in the Galaxy.
For the inverse Compton (IC) component of the gamma-ray emission, 
there is a significant uncertainty in the interstellar radiation field density near the GC \citep[e.g.,][]{2017ApJ...846...67P} in addition to 
the uncertainties of the CR distribution.
As a result, the agreement of the gamma-ray data with Galactic propagation models, such as GALPROP \citep{2007ARNPS..57..285S}, 
is rather poor in the GP, even after adjusting the components in energy bins and in Galactic rings.

The main goal of this paper is an analysis of the FB at low latitudes to estimate the uncertainties on the gamma-ray 
emission due to modeling of the Galactic foreground components.
In view of future searches with neutrino and Cherenkov telescopes, 
we will be primarily interested in morphology and spectrum of the FB at energies from 10 GeV to 1 TeV,
where Galactic components of emission are less significant compared energies below 10 GeV due to softer 
spectra of the Galactic components.

become less bright
and the corresponding uncertainties are less significant than at energies below 10 GeV.

We will use several methods to determine the Galactic foreground emission at high energies,
which we will use in the study of the FB.
At first, since the FB emission shows an anisotropy relative to the GC, 
while the foreground emission is expected to be symmetric (up to asymmetries in the gas distribution),
as a preliminary calculation, we find the difference in the gamma-ray data East and West of the GC 
(we also mask bright point sources).
Second, we use the fact that the spectrum of the FB is harder than the spectra of the other components of Galactic emission
above $\sim$ 1 GeV.
Consequently, the contribution of the FB at low energies near the GP is relatively small. 
In the second method we use the data $\lesssim$ 1 GeV as a template of Galactic foreground emission, 
which we fit to data at energies above 1 GeV.
The residual emission is used to estimate the emission from the FB.
At last, we compare the results with a prediction of GALPROP model of Galactic gamma-ray emission.





